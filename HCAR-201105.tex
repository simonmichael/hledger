% http://www.haskell.org/pipermail/haskell/2011-April/022720.html
% http://haskell.org/communities/05-2011/template.tex
% http://haskell.org/communities/05-2011/hcar.sty

\documentclass{scrreprt}
\usepackage{paralist}
\usepackage{graphicx}
\usepackage[final]{hcar}

%include polycode.fmt

\def\EMailRepl#1#2{\DoEMailRepl{#2} #1@@\End}
\def\DoEMailRepl#1#2@@#3\End
  {\if!#3!#2\else\DoDoEmailRepl{#1}{#2}#3\fi}
\def\DoDoEmailRepl#1#2#3@@{#2#1#3}
\def\EAt{ @@ }

\begin{document}

\begin{hcarentry}{hledger}
\report{Simon Michael}
\status{Ongoing development; suitable for daily use}
\participants{}% optional
\makeheader

hledger is a haskell port and friendly fork of John Wiegley's ledger.  It
is a robust command-line accounting tool with a simple human-editable data
format. Given a plain text file describing transactions, of money or any
other commodity, hledger will print the chart of accounts, account
balances, or transactions you're interested in.  It can also help you
record transactions, or convert CSV data from your bank. There are also
curses and web interfaces. The project aims to provide a reliable,
practical day-to-day financial reporting tool, and also a useful library
for building financial apps in haskell.

Since hledger's last HCAR entry in 2009, hledger became cabalised, had 10
non-bugfix releases on hackage, split into multiple packages, acquired a
public mailing list, bug tracker, fairly comprehensive manual,
cross-platform binaries, and has grown to 5k lines of code and 15
committers. 0.14 has just been released, with 5 code committers.

The project is available under the GNU GPLv3 or later, at http://hledger.org .

Current plans are to continue development at a steady pace, to attract
more developers, and to become more useful to a wider range of users, eg
by building in more awareness of standard accounting procedures and by
improving the web and other interfaces.

\FurtherReading
  \url{http://hledger.org}
\end{hcarentry}

\end{document}
